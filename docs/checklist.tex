\documentclass[a4paper,12pt]{article}

\usepackage[margin=2cm]{geometry} % 'normal' margin
\usepackage[T1]{fontenc} % encoding in the pdf, T1 for french
\usepackage[utf8]{inputenc} % encoding of this file
\usepackage{fancyhdr} % custom headers
\usepackage{hyperref} % hyperlink/clickable
\usepackage{amssymb} % we need the square symbol
\usepackage{enumitem} % custom enumeration
\newlist{checklist}{itemize}{2} % max-depth 2
\setlist[checklist]{label=$\square$}

%% Document properties
\def\mytitle{Group project submission instructions}\title{\mytitle}
\def\mydate{Mar. 2017}\date{\mydate}
\def\myauthor{LINGI2142}\author{\myauthor}

%% Headers
\lfoot{} % nothing in bottom left
\cfoot{} % nothing in bottom center
\rfoot{\thepage} % page in bottom right
\lhead{% course id in top left
    \large
    \href{https://moodleucl.uclouvain.be/course/view.php?id=9209
         }{\myauthor-- Computer networks: configuration and management}
}
\chead{} % nothing in top center
\rhead{% date in top right
    \large\mydate%
}

\begin{document}
\rmfamily % roman font
\pagestyle{fancy} % Custom header on all pages

\begin{center}
    \huge\bfseries\mytitle%
    \vspace{1ex}
\end{center} % title

\section{Deliverables}
The project submission is split in four deliverables.
\begin{description}
    \item[Archive of the project] We expect to receive
        an \textbf{archive} containing \textbf{every script} used to start the network (this
        includes all provided scripts), to test it. It should also contain a
        short README indicating how to start the network, and how should the
        test scripts be run and what should be expected from them.
        You \textbf{must comment/justify} every part of your \textbf{configuration
        files}, as well as mention the \textbf{default values} used if you rely
        on them.
    \item[Project report] We expect a short report (.pdf's only), containing:
        \begin{itemize}
            \item \textbf{A per-group introduction}, which presents the addressing plan
                used in the network and its rationale, topology
                changes, \dots 
            \item \textbf{Individual sections with author names}
                on the individual design tasks. These sections should focus on
                the high-level objectives that were implemented (e.g.\ security
                policy, queuing tree, combination of load-balancing
                techniques), their actual implementation overview, the tests
                that were realized to ensure the correctness of the
                implementation as well as a discussion on the extensibility of
                the approach (e.g.\ steps to add a new router/server/host).
        \end{itemize}
        Do not forget to \textbf{provide your sources} towards the readings you made in
        order to discover the usual practice when configuring a network!
    \item[Individual reviews] Each group will receive the projects of two other
        groups. Each group member then will have to \textbf{write a review on} both
        projects on \textbf{the same task that he implemented} in his project (or on the
        introduction/addressing plan if the group did not do it). E.g.\ a
        student that implemented routing will write 2 reviews on routing; a
        student that implemented monitoring might write 1 review on monitoring
        and 1 on the introduction/addressing plan.
    \item[Changelog] Based on the received reviews, each group will have time
        to improve some aspect of its project. Each \textbf{change must be
        documented and justified} in a global changelog for the project (a file
        describing what has been changed, how, and why). The final version of
        your project will consist on an archive of your project script, as well
        as this changelog.
\end{description}

\section{Submission schedule}
All submissions and reviews will be made on HotCRP
(\url{https://hotcrp.info.ucl.ac.be/LINGI2142/}). Please create an account
there using you \textbf{@student.uclouvain.be} email address! 
\begin{description}
    \item[20/03--First version of the project and report] There should only be
        \textbf{one submission per group}. The submission will be opened
        until 11.59am. Make sure to enter all author names and student email
        addresses (as this will create accounts on the fly if not yet done).
    \item[01/04--Individual reviews] We will notify you when the reviews have
        been assigned. \textbf{Each group member} will encode its own reviews
        on \textbf{HotCRP}. You will receive an email containing the needed
        login information.
    \item[01/05--Final version of the project and changelog] As before, only
        \textbf{one submission for the whole group}, but this this on
        \textbf{Moodle}.
\end{description}

\section{Grading}
The project is worth 40\% of the course grades. These are allocated as:
\begin{itemize}
    \item Individual review: 10\%. We will rate whether your reviews are
        consistent with the project itself (i.e.\ don't say `this project is
        terrible' if this is not the case), whether you backed up all claims, how
        complete were you in the review process (did you read the report? Read
        the relevant configuration files? Executed some tests of your own?
        \dots)
    \item Individual task: 15\%. We will rate your individual section in the
        report as well as your actual implementation and generated
        configuration files.
    \item Group: 15\%. This an average of every group member individual task
        grade, as well as an appreciation of the group section in the report.
\end{itemize}

\section{Sample feature sets}
Below are examples features that should be implemented in your network to get
it to run properly, as well as be tested
(i.e.\ each feature should have an associated test script)!

\subsection{Routing}
\begin{checklist}
    \item Your intra-domain routing protocol distributes routes for both
        prefixes assigned to your network.
    \item You export the appropriate prefix to each provider over BGP.\@
    \item Every host/router/server inside the network can communicate with 
        every other node in the network, and has internet access, provided the 
        security policies allow it.
    \item Your network can survive any link or router failure automatically
        (test this by bringing interface down on routers or by killing all
        routing daemons on a given router).
\end{checklist}

\subsection{Security}
\begin{checklist}
    \item You implemented policies that forbid some traffic from leaving the
        network (e.g.\ cameras) or enter it (e.g.\ RPF check, incoming
        connections towards non-servers) or instead white list it (e.g.\ DNS
        queries).
    \item You implemented access-control policies inside the network to
        explicitly allow or deny some connections (e.g.\ student-to-student
        connections, access to routers)
    \item You rules take into account the fact that hosts are dual-homed, and
        use dynamic addresses.
    \item You properly restrict the scope of some traffic (e.g.\ routing
        protocol messages should never be seen on non router-to-router links).
\end{checklist}

\subsection{End-user management}
\begin{checklist}
    \item Your DNS server acts as forwarder toward \texttt{fd00:{}:d} and is
        authoritative on the zone \texttt{groupX.ingi}.
    \item All routers/servers should have an associated AAAA/PTR record for
        their \texttt{groupX.ingi} FQDN.\@
    \item Non-infrastructure hosts receive automatically multiple IPv6
        addresses, information on where is the DNS server, and default routes.
        (define in
        the configuration file the user-type, thus provision statically the
        correct VLAN).
    \item Your setup is resilient to any single link or node failure (i.e.\
        you have at least 2 instances of everything at separate locations).
\end{checklist}

\subsection{Services}
\begin{checklist}
    \item Your configured SSH servers on multiple hosts (at least on all
        routers/servers), with different keys, and specific authorized\_keys
        file.
    \item Your network has dual-homed web servers serving static pages.
    \item Your web servers are behind a load-balancer.
    \item Your setup is resilient to any single link or node failures (i.e.\
        you have at least 2 instances of everything at separate locations).
\end{checklist}

\subsection{QoS}
\begin{checklist}
    \item You provide some guarantees on the QoS applied to traffic depending
        on its class (e.g.\ mission critical traffic is never dropped nor
        queued, inter data-center traffic will not never be dropped be can be
        queued, traffic towards the commercial internet has a best-effort
        service).
    \item Your queuing policies never cause a traffic flow to be shaped/policed
        if there is free bandwidth available (i.e.\ if theres is only one active
        users, it should be able to use the full network bandwidth).
    \item Your queuing policies handle both burst traffic (i.e.\ HTTP
        request/response) as well a long-lived flow (i.e.\ file transfer).
    \item Your queuing policies try to minimize starvation where possible
        (i.e.\ if two flows have the same class/priority, both should get
        \emph{some} bandwidth--even if they both cause their class to exceed
        its reserved bandwidth).
\end{checklist}

\subsection{Monitoring}
\begin{checklist}
    \item You configured a monitoring infrastructure, with dedicated servers.
    \item You collect on those servers statistics from the routers (e.g.\ SNMP
        counters, netflow data), logs from the servers/routers (e.g.\ program
        logs, syslog).
    \item You have scripts on the monitoring server that can report on the
        status of services in the network (e.g.\ list whether all service
        instances are alive, their current load, firewall alerts/drop counters)
    \item Your setup is resilient to any single link or node failures (i.e.\
        you have at least 2 instances of everything at separate locations).
\end{checklist}

\end{document}
